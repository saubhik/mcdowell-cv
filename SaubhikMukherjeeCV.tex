%! TeX program = lualatex

\documentclass[]{mcdowellcv}

\name{
  \href{https://www.linkedin.com/in/saubhikm/} {Saubhik Mukherjee}
}
\address{
  1735 Woodland Ave, Apt 62 \linebreak East Palo Alto, CA 94303 \linebreak
  United States
}
\contacts{
  Mobile: (470) 313-0534 \linebreak \ULurl{saubhik.mukherjee@gmail.com}
}

\begin{document}
  \makeheader

  \begin{cvsection}{Employment}
    \begin{cvsubsection}
      {Senior Software Engineer \linebreak \textit{Machine Learning (ML) compiler}}
      {\href{https://sambanova.ai/}{SambaNova Systems, Inc. \linebreak Palo Alto, CA, USA}}
      {May 23, 2022 - Present}
        \begin{itemize}
          \item
            Build \href{https://mlir.llvm.org/}{MLIR} and
            \href{https://llvm.org/}{LLVM}-based compiler layers and
            compiler tools to transform, optimize, debug, and execute ML
            models on proprietary ML accelerator architectures.
          \item
            Build scalable and high-quality production compiler
            infrastructure using well-established and emerging techniques
            and push the boundaries of compiler design.
          \item
            Develop, maintain, and debug compiler optimization algorithms
            on ML graphs and add compiler support for new hardware
            architectures.
          \item
            Analyze and improve compile-time and run-time performance
            across multiple AI hardware architectures and ML frameworks,
            such as \href{https://www.tensorflow.org/}{TensorFlow} and
            \href{https://pytorch.org/}{PyTorch}, to support new
            state-of-the-art training and inference.
          \item
            Collaborate with ML researchers and engineers to guide compiler
            development for future ML trends.
          \item
            \textit{Tools}: \href{https://isocpp.org/}{C++},
            \href{https://clang.llvm.org/}{Clang},
            \href{https://mlir.llvm.org/}{MLIR},
            \href{https://lldb.llvm.org/}{LLDB},
            \href{https://cmake.org/}{cmake},
            \href{https://ninja-build.org/}{ninja},
            \href{https://gperftools.github.io/gperftools/cpuprofile.html}{gperf},
            \href{https://github.com/tmux/tmux}{tmux},
            \href{https://neovim.io/}{neovim},
            \href{https://github.com/universal-ctags/ctags}{ctags},
            \href{https://clangd.llvm.org/}{clangd},
            \href{https://www.synopsys.com/verification/simulation/vcs.html}{Synopsys
            VCS}.
        \end{itemize}
    \end{cvsubsection}

    \begin{cvsubsection}
      {Software Engineer \linebreak \textit{Machine Learning (ML) systems}}
      {\href{https://www.alpaca.markets/jp/about.html}{AlpacaJapan, Co. Ltd. \linebreak Tokyo, Japan}}
      {Jan 1, 2019 - Jan 8, 2021}
        \begin{itemize}
          \item
            Design, develop, maintain, and test live production software
            systems for delivering stock price predictions.
          \item
            Collaborate with data science \& engineering team to integrate
            different software systems and deploy and upgrade ML models in
            live production financial forecasting software.
          \item
            Handle installation \& maintenance of new data sources and
            develop the data platform used for ML model R\&D.
          \item
            Collect and document client requirements for future releases and
            make extensible and robust software design decisions for
            developing server \& client web applications; responsible for
            10\% annual revenue growth.
          \item
            Manage software releases with an agile mindset and develop
            workflows for fast production recovery in case of failures.
          \item
            Driving innovation by evaluating new technologies, original
            financial data sources, and recent research papers that add value
            to Alpaca's products.
          \item
            \textit{Tools}: \href{https://www.python.org/}{Python},
            \href{https://react.dev/}{React},
            \href{https://developer.mozilla.org/en-US/docs/Web/JavaScript}{JavaScript},
            \href{https://flask.palletsprojects.com/}{Flask},
            \href{https://www.postgresql.org/}{PostgreSQL},
            \href{https://www.sqlalchemy.org/}{SQLAlchemy},
            \href{https://alembic.sqlalchemy.org/en/latest/}{Alembic},
            \href{https://kubernetes.io/}{Kubernetes},
            \href{https://www.docker.com/}{Docker},
            \href{https://pytorch.org/}{PyTorch},
            \href{https://pandas.pydata.org/}{Pandas},
            \href{https://numpy.org/}{NumPy}, \href{https://scipy.org/}{SciPy},
            \href{https://github.com/spotify/luigi}{Luigi},
            \href{https://circleci.com/}{CircleCI},
            \href{https://argoproj.github.io/cd/}{Argo CD},
            \href{https://auth0.com/}{Auth0},
            \href{https://www.datadoghq.com/}{Datadog}.
        \end{itemize}
    \end{cvsubsection}

    \begin{cvsubsection}
      {Data Scientist \linebreak \textit{ML in pricing research}}
      {\href{https://www.abinbev-india.com/}{Anheuser-Busch InBev \linebreak Bangalore, India}}
      {Jun 19, 2017 - Dec 28, 2018}
        \begin{itemize}
          \item
            Develop machine learning models to estimate ABInBev's market
            share and revenue in different pricing scenarios of beer SKUs
            across multiple countries, using both
            \href{https://www.r-project.org/about.html}{R} statistical
            programming language and \href{https://www.python.org/}{Python}.
          \item
            Conduct extensive experiments to determine the significant
            variables in ML models and create automated scripts to replicate
            the process, using \href{https://keras.io/}{Keras},
            \href{https://www.tensorflow.org/}{TensorFlow}.
          \item
            Create pricing conjoint survey questionnaires and handle data
            management and pre-processing using customized scripts and
            workflows; used \href{https://dplyr.tidyverse.org/}{dplyr},
            \href{https://tidyr.tidyverse.org/}{tidyr}.
          \item
            Interact and collaborate with business heads in different
            countries to include different pricing scenarios in conjoint
            based on the business requirements and present the pricing
            analysis results for business actions; used
            \href{https://ggplot2.tidyverse.org/}{ggplot2}.
          \item
            Develop various optimization algorithms based on pricing analysis
            results to maximize the business objective, such as market share
            or revenue; used
            \href{https://cran.r-project.org/web/packages/nloptr/}{nloptr}.
          \item
            Create UI dashboards that display conjoint analysis results for
            business to gain actionable insights, using
            \href{https://shiny.rstudio.com/}{Shiny},
            \href{https://posit.co/}{RStudio}.
        \end{itemize}
    \end{cvsubsection}
  \end{cvsection}

  \begin{cvsection}{Internships \& Research}
    \begin{cvsubsection}
      {Graduate Research Assistant}
      {\href{https://scs.gatech.edu/}{Georgia Tech, Atlanta}}
      {Aug 2021 - May 2022}
        \begin{itemize}
            \item Ported Facebook's QUIC implementation, \href{https://github.com/facebookincubator/mvfst}{mvfst}, to rely on the efficient kernel-bypass network stack (threading \& socket) APIs provided by MIT's \href{https://www.usenix.org/conference/nsdi19/presentation/ousterhout}{Shenango} and achieve low tail latency \& increase CPU efficiency.
            \item \href{https://github.com/saubhik/caladan/pulls}{\textit{QuicNIC}}: Offloaded GSO \& crypto (encryption, decryption) to a dedicated CPU core to obtain record QUIC throughputs (x5). \textit{Skills}:
            \href{https://quicwg.org/}{QUIC},
            \href{https://github.com/shenango/caladan}{caladan}, \href{https://github.com/facebookincubator/mvfst}{mvfst}, \href{https://github.com/facebook/folly}{folly}, \href{https://github.com/facebookincubator/fizz}{fizz}, profiling prod C++ codebase, \href{https://www.brendangregg.com/FlameGraphs/cpuflamegraphs.html}{CPU FlameGraphs}.
        \end{itemize}
    \end{cvsubsection}

    \begin{cvsubsection}{Linux Contributor, Intern}{\href{https://summerofcode.withgoogle.com/archive/2021/projects/4818588170452992}{Google Summer of Code}}{May '21 - Aug '21}
        \begin{itemize}
            \item Analyze and fix race condition bugs in the Linux kernel 5.4 device drivers based on software verification static analysis tool, \href{https://forge.ispras.ru/projects/klever}{Klever}. Part of \textit{Linux Standards Base} \& \textit{Linux Driver Verification}.
            \item Accepted \href{https://lore.kernel.org/lkml/?q=saubhik}{patches} to kernel mainline. \textit{Skills}: \href{https://www.kernel.org/}{Linux kernel development}, C.
        \end{itemize}
    \end{cvsubsection}
  \end{cvsection}

  \begin{cvsection}{Education}
    \begin{cvsubsection}{Atlanta, GA}{\href{https://www.gatech.edu/}{Georgia Institute of Technology}}{Jan '20 - May '22}
      \begin{itemize}
        \item \textbf{Master of Science in Computer Science} with \textit{Systems Specialization}, May 2022. \textbf{GPA: 4.0}
        \item \textit{Courses}: Operating Systems; Computer Architecture; Compilers; Networks; Distributed Systems; Databases; HPC; Algorithms.
      \end{itemize}
    \end{cvsubsection}
    \begin{cvsubsection}{Kolkata, IN}{\href{https://www.isical.ac.in/}{Indian Statistical Institute}}{Jul '15 - Jun '17}
      \begin{itemize}
        \item \textbf{Master of Science in Quantitative Economics}; full scholarship \& monthly stipends
        \item \textit{Relevant Courses}: Optimization; Game Theory.
      \end{itemize}
    \end{cvsubsection}
    \begin{cvsubsection}{Chennai, IN}{\href{https://www.cmi.ac.in/}{Chennai Mathematical Institute}}{Aug '12 - Apr '15}
      \begin{itemize}
        \item \textbf{Bachelor of Science in Mathematics and Computer Science}, \href{https://www.online-inspire.gov.in/}{Innovation in Science Pursuit Scholar}
        \item \textit{Courses}: Algorithms; Programming Languages; Discrete Math; Theory of Computation; Logic.
      \end{itemize}
    \end{cvsubsection}
  \end{cvsection}
\end{document}
