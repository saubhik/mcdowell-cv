%! TeX program = lualatex

\documentclass[]{mcdowellcv}

\name{
  \href{https://www.linkedin.com/in/saubhikm/} {Saubhik Mukherjee}
}
\address{
  1735 Woodland Ave, Apt 62 \linebreak East Palo Alto, CA 94303 \linebreak
  United States
}
\contacts{
  +1 (470) 313-0534 \linebreak \ULurl{saubhik.mukherjee@gmail.com}
}

\begin{document}
  \makeheader

  \begin{cvsection}{Employment}
    \begin{cvsubsection}
      {Senior Software Engineer \linebreak \textit{Machine Learning (ML) compiler}}
      {\href{https://sambanova.ai/}{SambaNova Systems, Inc. \linebreak Palo Alto, CA, USA}}
      {May 23, 2022 - Present}
        \begin{itemize}
          \item
            Build \href{https://mlir.llvm.org/}{MLIR} and
            \href{https://llvm.org/}{LLVM}-based compiler layers and
            compiler tools to transform, optimize, debug, and execute ML
            models on proprietary ML accelerator architectures.
          \item
            Build scalable and high-quality production compiler
            infrastructure using well-established and emerging techniques
            and push the boundaries of compiler design.
          \item
            Develop, maintain, and debug compiler optimization algorithms
            on ML graphs and add compiler support for new hardware
            architectures.
          \item
            Analyze and improve compile-time and run-time performance
            across multiple AI hardware architectures and ML frameworks,
            such as \href{https://www.tensorflow.org/}{TensorFlow} and
            \href{https://pytorch.org/}{PyTorch}, to support new
            state-of-the-art training and inference.
          \item
            Collaborate with ML researchers and engineers to guide compiler
            development for future ML trends.
          \item
            \textit{Tools}: \href{https://isocpp.org/}{C++},
            \href{https://clang.llvm.org/}{Clang},
            \href{https://mlir.llvm.org/}{MLIR},
            \href{https://lldb.llvm.org/}{LLDB},
            \href{https://cmake.org/}{cmake},
            \href{https://ninja-build.org/}{ninja},
            \href{https://gperftools.github.io/gperftools/cpuprofile.html}{gperf},
            \href{https://github.com/tmux/tmux}{tmux},
            \href{https://neovim.io/}{neovim},
            \href{https://github.com/universal-ctags/ctags}{ctags},
            \href{https://clangd.llvm.org/}{clangd},
            \href{https://www.synopsys.com/verification/simulation/vcs.html}{Synopsys
            VCS}.
        \end{itemize}
    \end{cvsubsection}

    \begin{cvsubsection}
      {Software Engineer \linebreak \textit{Machine Learning (ML) systems}}
      {\href{https://www.alpaca.markets/jp/about.html}{AlpacaJapan, Co. Ltd. \linebreak Tokyo, Japan}}
      {Jan 1, 2019 - Jan 8, 2021}
        \begin{itemize}
          \item
            Design, develop, maintain, and test live production software
            systems for delivering stock price predictions.
          \item
            Collaborate with data science \& engineering team to integrate
            different software systems and deploy and upgrade ML models in
            live production financial forecasting software.
          \item
            Handle installation \& maintenance of new data sources and
            develop the data platform used for ML model R\&D.
          \item
            Collect and document client requirements for future releases and
            make extensible and robust software design decisions for
            developing server \& client web applications; responsible for
            10\% annual revenue growth.
          \item
            Manage software releases with an agile mindset and develop
            workflows for fast production recovery in case of failures.
          \item
            Driving innovation by evaluating new technologies, original
            financial data sources, and recent research papers that add value
            to Alpaca's products.
          \item
            \textit{Tools}: \href{https://www.python.org/}{Python},
            \href{https://react.dev/}{React},
            \href{https://developer.mozilla.org/en-US/docs/Web/JavaScript}{JavaScript},
            \href{https://flask.palletsprojects.com/}{Flask},
            \href{https://www.postgresql.org/}{PostgreSQL},
            \href{https://www.sqlalchemy.org/}{SQLAlchemy},
            \href{https://alembic.sqlalchemy.org/en/latest/}{Alembic},
            \href{https://kubernetes.io/}{Kubernetes},
            \href{https://www.docker.com/}{Docker},
            \href{https://pytorch.org/}{PyTorch},
            \href{https://pandas.pydata.org/}{Pandas},
            \href{https://numpy.org/}{NumPy}, \href{https://scipy.org/}{SciPy},
            \href{https://github.com/spotify/luigi}{Luigi},
            \href{https://circleci.com/}{CircleCI},
            \href{https://argoproj.github.io/cd/}{Argo CD},
            \href{https://auth0.com/}{Auth0},
            \href{https://www.datadoghq.com/}{Datadog}.
        \end{itemize}
    \end{cvsubsection}

    \begin{cvsubsection}
      {Data Scientist \linebreak \textit{ML in pricing research}}
      {\href{https://www.abinbev-india.com/}{Anheuser-Busch InBev \linebreak Bangalore, India}}
      {Jun 19, 2017 - Dec 28, 2018}
        \begin{itemize}
          \item
            Develop machine learning models to estimate ABInBev's market
            share and revenue in different pricing scenarios of beer SKUs
            across multiple countries, using both
            \href{https://www.r-project.org/about.html}{R} statistical
            programming language and \href{https://www.python.org/}{Python}.
          \item
            Conduct extensive experiments to determine the significant
            variables in ML models and create automated scripts to replicate
            the process, using \href{https://keras.io/}{Keras},
            \href{https://www.tensorflow.org/}{TensorFlow}.
          \item
            Create pricing conjoint survey questionnaires and handle data
            management and pre-processing using customized scripts and
            workflows; used \href{https://dplyr.tidyverse.org/}{dplyr},
            \href{https://tidyr.tidyverse.org/}{tidyr}.
          \item
            Interact and collaborate with business heads in different
            countries to include different pricing scenarios in conjoint
            based on the business requirements and present the pricing
            analysis results for business actions; used
            \href{https://ggplot2.tidyverse.org/}{ggplot2}.
          \item
            Develop various optimization algorithms based on pricing analysis
            results to maximize the business objective, such as market share
            or revenue; used
            \href{https://cran.r-project.org/web/packages/nloptr/}{nloptr}.
          \item
            Create UI dashboards that display conjoint analysis results for
            business to gain actionable insights, using
            \href{https://shiny.rstudio.com/}{Shiny},
            \href{https://posit.co/}{RStudio}.
        \end{itemize}
    \end{cvsubsection}
  \end{cvsection}

  \begin{cvsection}{Internships \& Research}
    \begin{cvsubsection}
      {Graduate Research Assistant \linebreak \textit{Networked Systems}}
      {\href{https://scs.gatech.edu/}{Georgia Institute of Technology \linebreak Atlanta, GA, USA}}
      {Aug 23, 2021 - May 7, 2022}
        \begin{itemize}
          \item
            Developed
            \href{https://github.com/saubhik/caladan/pulls}{\textit{QuicNIC}},
            a software NIC to accelerate the \href{https://quicwg.org/}{QUIC}
            stack, to move segmentation, pacing, and encryption from the
            application to a dedicated core implementing software NIC
            functionality; achieved TCP-like single-connection throughput.
          \item
            Ported Meta's production QUIC (network protocol) implementation,
            \href{https://github.com/facebookincubator/mvfst}{mvfst} (\&
            dependencies \href{https://github.com/facebook/folly}{folly},
            \href{https://github.com/facebookincubator/fizz}{fizz}), to rely on
            the efficient kernel-bypass network stack (custom threading \&
            socket libraries) APIs provided by MIT's
            \href{https://www.usenix.org/conference/nsdi19/presentation/ousterhout}{Shenango}
            (\href{https://github.com/shenango/caladan}{caladan}) and achieve
            low tail latency \& increase CPU efficiency; involved CPU profiling
            using
            \href{https://www.brendangregg.com/FlameGraphs/cpuflamegraphs.html}{flame
            graphs}.
          \item
            This
            \href{https://drive.google.com/file/d/1-y7gsG67KGIeD2qhomVT0vTPlPB4Vep1/view?usp=sharing}{research}
            was supervised by professors
            \href{https://saeed.github.io/index.html}{Ahmed Saeed} and
            \href{https://www.cc.gatech.edu/fac/Mostafa.Ammar/}{Mostafa Ammar}.
        \end{itemize}
    \end{cvsubsection}

    \begin{cvsubsection}
      {Open Source Contributor \linebreak \textit{The Linux Foundation}}
      {\href{https://summerofcode.withgoogle.com/archive/2021/projects/4818588170452992}{Google Summer of Code}}
      {Jun 7, 2021 - Aug 23, 2021}
        \begin{itemize}
          \item
            Analyze and fix race condition bugs in the Linux Kernel 5.4 device
            drivers based on software verification static analysis tool,
            \href{https://forge.ispras.ru/projects/klever}{Klever}. Accepted
            \href{https://lore.kernel.org/lkml/?q=saubhik}{patches} to kernel
            mainline. \textit{Skills}: \href{https://www.kernel.org/}{Linux
            kernel development}, C.
        \end{itemize}
    \end{cvsubsection}
  \end{cvsection}

  \begin{cvsection}{Education}
    \begin{cvsubsection}
      {Master of Science in Computer Science, \textit{Systems Specialization}}
      {\href{https://www.gatech.edu/}{Georgia Institute of Technology \linebreak Atlanta, GA, USA}}
      {Jan 6, 2020 - May 7, 2022}
      \begin{itemize}
        \item
          \textit{Major}: Computer Science, \textit{Concentration}: Computing
          Systems. \textit{Overall GPA}: 4.00.
          \href{https://drive.google.com/file/d/1rG1vO2I3fK_aVtkboU8p6I_6vYUQBlf9/view?usp=sharing}{Transcript
          link}.
        \item
          \textit{Courses}: High Performance Computer Architecture, Computer
          Networks, Compilers \& Interpreters, Advanced Operating Systems, High
          Performance Computing, Database Systems Concepts \& Design,
          Distributed Computing, Introduction to Graduate Algorithms,
          Datacenter Networks \& Systems.
      \end{itemize}
    \end{cvsubsection}
    \begin{cvsubsection}
      {Master of Science in Quantitative Economics}
      {\href{https://www.isical.ac.in/}{Indian Statistical Institute \linebreak Kolkata, India}}
      {Jul 2015 - Jun 2017}
      \begin{itemize}
        \item
          Passed in First Division with Distinction.
          \href{https://drive.google.com/file/d/1b0q9SZj-vi8_KwSI1X1aQ1qtAroeXgw0/view?usp=sharing}{Transcript
          link}.
        \item
          \textit{Courses}: Microeconomics I, Game Theory I, Statistics,
          Computer Programming and Applications, Basic Economics, Microeconomic
          Theory II, Macroeconomic Theory I, Econometric Methods I, Time Series
          Analysis and Forecasting, Game Theory II, Macroeconomics II,
          Econometric Applications I, Individual and Collective Choice,
          Political Economy, Econometric Applications II, Economic Development,
          Bayesian Econometrics, Auction Theory, Optimization Techniques.
      \end{itemize}
    \end{cvsubsection}
    \begin{cvsubsection}
      {Bachelor of Science (Honours) in Mathematics \& Computer Science}
      {\href{https://www.cmi.ac.in/}{Chennai Mathematical Institute \linebreak Chennai, India}}
      {Aug 3, 2012 - Jul 25, 2015}
      \begin{itemize}
        \item
          \textit{Cumulative GPA}: 06.93 (out of 10).
          \href{https://drive.google.com/file/d/1azLzpG9HNLKuCWK21QHGjXc02za2fGW3/view?usp=sharing}{Transcript
          link}.
        \item
          \textit{Courses}: Algebra I, Calculus I, Classical Mechanics I,
          English, Introduction to Programming, Advanced Programming, Algebra
          II, Calculus II, Discrete Mathematics, Probability Theory,
          Environment Course, Algebra III, Calculus III, Design \& Analysis of
          Algorithms, Real Analysis, Theory of Computation, Complex Analysis,
          Differential Equations, Programming Language Concepts, Topology,
          Differential Geometry, Algebra IV, Introduction to Logic, Commutative
          Algebra, Optimization Techniques, Economics, Electrodynamics I,
          Finance, Representation Theory.
      \end{itemize}
    \end{cvsubsection}
  \end{cvsection}
\end{document}
