%! TeX program = lualatex

\documentclass[]{mcdowellcv}

\name{
  \href{https://www.linkedin.com/in/saubhikm/}{Saubhik Mukherjee} \linebreak
  \small {AI Compiler Engineer}
}

\address{
  433 Jarvis Street \linebreak
  Toronto, ON M4Y 2G9 \linebreak 
  Canada (Permanent Resident)
}
\contacts{
  USA: +1 (470) 313-0534 \linebreak 
  CAN: +1 (226) 336-3865 \linebreak
  \ULurl{saubhik@gatech.edu}
}

\begin{document}
\makeheader

\begin{cvsection}{Employment}
  \begin{cvsubsection}
    {Senior Software Engineer \linebreak \textit{Machine Learning (ML) compiler}}
    {\href{https://sambanova.ai/}{SambaNova Systems, Inc. \linebreak Palo Alto, CA, USA}}
    {May 23, 2022 - Present}
    \begin{itemize}
      \item
            Built \href{https://mlir.llvm.org/}{MLIR} \&
            \href{https://llvm.org/}{LLVM}-based compiler layers \& tools to
            transform, optimize, debug, execute state-of-the-art AI models on
            proprietary (non-Von Neumann, dataflow-based) ML accelerator
            architectures; standardized compiler infrastructure to support for
            rapid new model bringup (10 per month) with industry-leading model
            performance
      \item
            Engineered and refined an embedded DSL abstracting memory \&
            compute units, ensuring robustness and adaptability to evolving
            accelerator architectures and emerging ML models; designed
            optimizations to improve resource count by 5\%
      \item
            Analyzed \& improved compile-time \& run-time performance across
            multiple AI hardware architectures \& ML frameworks, such as
            \href{https://www.tensorflow.org/}{TensorFlow} \&
            \href{https://pytorch.org/}{PyTorch}; achieved a 20\% reduction in
            compilation time for large AI models, resulting in faster model
            iteration \& development cycles
            % TODO: Need more impactful points
      \item
            \textit{Tools}:
            \href{https://isocpp.org/}{C++},
            \href{https://clang.llvm.org/}{Clang},
            \href{https://mlir.llvm.org/}{MLIR},
            \href{https://lldb.llvm.org/}{LLDB},
            \href{https://cmake.org/}{cmake},
            \href{https://bazel.build/}{Bazel},
            \href{https://ninja-build.org/}{ninja},
            \href{https://gperftools.github.io/gperftools/cpuprofile.html}{gperf},
            \href{https://github.com/tmux/tmux}{tmux},
            \href{https://neovim.io/}{neovim},
            \href{https://github.com/universal-ctags/ctags}{ctags},
            \href{https://clangd.llvm.org/}{clangd},
            \href{https://www.synopsys.com/verification/simulation/vcs.html}{Synopsys VCS}
    \end{itemize}
  \end{cvsubsection}

  \begin{cvsubsection}
    {Software Engineer \linebreak \textit{Machine Learning (ML) systems}}
    {\href{https://www.alpaca.markets/jp/about.html}{AlpacaJapan, Co. Ltd. \linebreak Tokyo, Japan}}
    {Jan 1, 2019 - Jan 8, 2021}
    \begin{itemize}
      \item
            Designed, developed, maintained, \& tested production stock price
            prediction system, leading to 20\% improvement in model deployment
            efficiency; developed workflows for fast production recovery
            reducing downtime by 30\%
      \item
            Managed installation \& maintenance of new OHLCV data sources \&
            developed the data platform used for ML model R\&D, resulting in
            a 25\% reduction in data acquisition \& processing time
      \item
            Collected \& documented client requirements for future releases \&
            made extensible \& robust software design decisions for developing
            server \& client web apps, contributing to a 10\% annual revenue
            growth
      \item
            \textit{Tools}:
            \href{https://www.python.org/}{Python},
            \href{https://react.dev/}{React},
            \href{https://developer.mozilla.org/en-US/docs/Web/JavaScript}{JavaScript},
            \href{https://flask.palletsprojects.com/}{Flask},
            \href{https://www.postgresql.org/}{PostgreSQL},
            \href{https://www.sqlalchemy.org/}{SQLAlchemy},
            \href{https://alembic.sqlalchemy.org/en/latest/}{Alembic},
            \href{https://kubernetes.io/}{Kubernetes},
            \href{https://www.docker.com/}{Docker},
            \href{https://pytorch.org/}{PyTorch},
            \href{https://p\&as.pydata.org/}{P\&as},
            \href{https://numpy.org/}{NumPy}, \href{https://scipy.org/}{SciPy},
            \href{https://github.com/spotify/luigi}{Luigi},
            \href{https://circleci.com/}{CircleCI},
            \href{https://argoproj.github.io/cd/}{Argo CD},
            \href{https://auth0.com/}{Auth0},
            \href{https://www.datadoghq.com/}{Datadog}
    \end{itemize}
  \end{cvsubsection}

  \begin{cvsubsection}
    {Data Scientist \linebreak \textit{ML in pricing research}}
    {\href{https://www.abinbev-india.com/}{Anheuser-Busch InBev \linebreak Bangalore, India}}
    {Jun 19, 2017 - Dec 28, 2018}
    \begin{itemize}
      \item
            Developed ML (multinomial logit, neural nets) models to estimate
            ABInBev's market share \& revenue in different pricing scenarios of
            beer SKUs across multiple countries, using
            \href{https://www.r-project.org/about.html}{R} \&
            \href{https://www.python.org/}{Python}; developed game-theoretic
            techniques \& numerical optimization algorithms (Nelder-Mead, BFGS,
            Simulated Annealing) based on pricing analysis results to maximize
            business KPIs; created UI dashboards for consuming model reports,
            using \href{https://shiny.rstudio.com/}{Shiny},
            \href{https://posit.co/}{RStudio}
    \end{itemize}
  \end{cvsubsection}
\end{cvsection}

\begin{cvsection}{Internships \& Research}
  \begin{cvsubsection}
    {Graduate Research Assistant \linebreak \textit{Networked Systems}}
    {\href{https://scs.gatech.edu/}{Georgia Institute of Technology \linebreak Atlanta, GA, USA}}
    {Aug 23, 2021 - May 7, 2022}
    \begin{itemize}
      \item
            Developed
            \href{https://github.com/saubhik/caladan/pulls}{\textit{QuicNIC}},
            a software NIC to accelerate the \href{https://quicwg.org/}{QUIC}
            stack, to move segmentation, pacing, \& encryption from the
            application to a dedicated core implementing software NIC
            functionality; achieved TCP-like single-connection throughput.
      \item
            Ported Meta's production QUIC (network protocol) implementation,
            \href{https://github.com/facebookincubator/mvfst}{mvfst} (\&
            dependencies \href{https://github.com/facebook/folly}{folly},
            \href{https://github.com/facebookincubator/fizz}{fizz}), to rely on
            the efficient kernel-bypass network stack (custom threading \&
            socket libraries) APIs provided by MIT's
            \href{https://www.usenix.org/conference/nsdi19/presentation/ousterhout}{Shenango}
            (\href{https://github.com/shenango/caladan}{caladan}) \& achieve low
            tail latency \& increase CPU efficiency; involved CPU profiling
            using
            \href{https://www.brendangregg.com/FlameGraphs/cpuflamegraphs.html}{flame
              graphs};
            \href{https://drive.google.com/file/d/1-y7gsG67KGIeD2qhomVT0vTPlPB4Vep1/view?usp=sharing}{research
              doc}
    \end{itemize}
  \end{cvsubsection}

  \begin{cvsubsection}
    {Open Source Contributor \linebreak \textit{The Linux Foundation}}
    {\href{https://summerofcode.withgoogle.com/archive/2021/projects/4818588170452992}{Google Summer of Code}}
    {Jun 7, 2021 - Aug 23, 2021}
    \begin{itemize}
      \item
            Analyze \& fix race condition bugs in the Linux Kernel 5.4 device
            drivers based on software verification static analyzer checker,
            \href{https://forge.ispras.ru/projects/klever}{Klever}. Accepted
            \href{https://lore.kernel.org/lkml/?q=saubhik}{patches} to kernel
            mainline. \textit{Skills}: \href{https://www.kernel.org/}{Linux
              kernel development}, C.
    \end{itemize}
  \end{cvsubsection}
\end{cvsection}

\begin{cvsection}{Education}
  \hfill
  \begin{itemize}
    \item
          \textbf{Master of Science in Computer Science \hfill Jan 2020 - May 2022}
          \begin{itemize}
            \item
                  \textbf{\href{https://www.gatech.edu/}{Georgia Institute of Technology, USA}}
            \item
                  \textit{Major}: Computer Science, \textit{Concentration}: Computing Systems. \textit{GPA}: 4.00;
                  \href{https://drive.google.com/file/d/1rG1vO2I3fK_aVtkboU8p6I_6vYUQBlf9/view?usp=sharing}{transcript link}.
          \end{itemize}
    \item
          \textbf{Master of Science in Quantitative Economics \hfill Jul 2015 - Jun 2017}
          \begin{itemize}
            \item
                  \textbf{\href{https://www.isical.ac.in/}{Indian Statistical Institute, India}}
            \item
                  Passed in \textit{First Division with Distinction};
                  \href{https://drive.google.com/file/d/1b0q9SZj-vi8_KwSI1X1aQ1qtAroeXgw0/view?usp=sharing}{transcript link}
          \end{itemize}
    \item
          \textbf{Bachelor of Science in Mathematics \& Computer Science \hfill Aug 2012 - Jul 2015}
          \begin{itemize}
            \item
                  \textbf{\href{https://www.cmi.ac.in/}{Chennai Mathematical Institute, India}}
            \item
                  \textit{GPA}: 06.93 (out of 10); \href{https://drive.google.com/file/d/1azLzpG9HNLKuCWK21QHGjXc02za2fGW3/view?usp=sharing}{transcript
                    link}.
          \end{itemize}
  \end{itemize}
\end{cvsection}
\end{document}
